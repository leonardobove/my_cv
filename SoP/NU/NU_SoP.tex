%%%%%%%%%%%%%%%%%%%%%%%%%%%%%%%%%%%%%%%%%%%%%%%%%%%%
% SPDX-License-Identifier: BSD-3-Clause
% Copyright (c) 2020, Salman Ahmad Faris
% https://github.com/salfaris/EasyPS
% Copyright (c) 2024, Daize Dong
% https://github.com/DaizeDong/Easier-PS-and-SoP
%%%%%%%%%%%%%%%%%%%%%%%%%%%%%%%%%%%%%%%%%%%%%%%%%%%%

\documentclass{easier_sop}

% --- ESSAY DISPLAY SETTINGS ---
\SetDocumentTitle{Statement of Purpose}
\SetStudentName{Leonardo Bove}           % Your name
\SetProgramName{PhD in Computer Science}           % Program you're applying for
\SetUniversityName{Northwestern University}     % University name
\SetUniversityAbbr{\GetUniversityName} % University abbreviation (default as the university name if not set)

% CONTENT
\begin{document}
	\thispagestyle{firstpageheader} % Use the expanded header on the first page
	
	I have always been fascinated by how hardware enables computation. This interest led me to a technical high school in Italy specializing in Electronics and later to a B.Sc.\ in Electronic Engineering at the University of Pisa, where I finally understood what was happening beneath my fingers as I typed on a keyboard. Learning about Moore's law, and its eventual limits, made me wonder whether the evolution of classical computing was nearing its end. Discovering that quantum computing was a real and rapidly developing field, rather than a distant concept, was incredibly exciting and made me feel as though we are witnessing a revolution comparable to the era of the first electronic computers.
	
	After this discovery, I quickly realized that I wanted to contribute to this scientific and engineering challenge and help make quantum computers useful for humanity. As with the classical computing systems we rely on today, their transformative impact emerged only after widespread democratization. Achieving this for quantum technologies will require not only advances in qubit performance from physicists, but also the essential contributions of computer scientists and engineers in building the electrical and software infrastructure that makes these systems accessible and usable.
	
	This is why my academic interests focus on studying and designing new architectures and tools that can help us achieve scalability and the long-anticipated quantum advantage, starting from the Noisy Intermediate-Scale Quantum (NISQ) devices available today. While researchers continue to improve qubit coherence times and gate fidelities, progress toward fault-tolerant quantum computing also depends on optimizing how quantum circuits are mapped onto hardware and on devising new qubit arrangements that enable innovative quantum error-correction codes. My interests therefore include developing quantum-circuit compilers that are aware of hardware constraints and noise characteristics, thereby enabling optimization and error-mitigation techniques, as well as exploring architectures that integrate classical processing units (e.g., CPUs, GPUs) with Quantum Processing Units (QPUs) to support hybrid algorithms such as the Variational Quantum Eigensolver (VQE). I believe that a Ph.D.\ program in Computer Science is the most suitable path for addressing these questions, as it would provide the academic rigor and technical preparation needed to engage with this emerging computational revolution, along with invaluable guidance from experienced faculty members.
	
	In particular, I believe that significant and impactful research is being conducted in the ``Systems and Networking'' and ``Quantum Science and Engineering'' groups within the Computer Science Department at Northwestern's McCormick School of Engineering. I am especially interested in the work of Professor Kaitlin N.\ Smith and Professor Nikos Hardavellas, who are developing compilers that automatically embed quantum error-detection codes into high-level circuits and designing modular QPU architectures with cross-module hierarchical compilers to enable scalable quantum computing. I am eager to contribute to research in these areas and to help design efficient systems that enable reliable and scalable quantum computation.
	
	From my perspective, these academic questions require a comprehensive understanding of the entire system. Although my background is not purely in Computer Science, I believe I am a strong candidate thanks to my training in electronics, my coursework in software and computer engineering, and my focused experience in quantum computing. I discovered quantum computing at the end of my Bachelor's degree and dedicated my thesis to studying the control and readout of the transmon qubit, while independently learning quantum computation theory and algorithms. This work highlighted the crucial role of classical hardware in quantum systems and led me to pursue a Master's degree in Electronic Engineering to study qubit controllers in greater depth. During my thesis with the SQMS Center at Fermilab, I assembled an FPGA-based controller using QICK and developed Qubase, a high-level Python pulse sequencer that generates hardware-specific instructions from hardware-agnostic experiments. Using this tool, I was able to easily describe multi-qubit sequences for fast-decay detection of $T_1$ relaxation times, which were automatically mapped onto a frequency-multiplexed device where qubits share the same feed line, though the same sequences could be mapped to architectures with individually addressed qubits. These experiences solidified my interest in developing the middleware that bridges high-level abstractions and low-level qubit control, making quantum computers more accessible and scalable.
	
	While preparing this application, I am also employing the tools I developed as a graduate research fellow in my thesis supervisor's laboratory in Pisa, where I built the hardware and software infrastructure that makes our engineering school's first quantum computer remotely accessible to any researcher on the network through the intuitive Qubase interface.
	
	I am certain that I will use this Ph.D.\ in Computer Science to pursue a career in the quantum-computing industry. I believe that Northwestern University would be an excellent choice for my future development, thanks to its proximity to Chicago, which is rapidly becoming a major hub for quantum technologies through initiatives such as the Illinois Quantum and Microelectronics Park, the Chicago Quantum Exchange, and several growing startups, including Infleqtion. This geographical advantage, together with the region’s dense network of experts, would provide invaluable perspectives. Moreover, my academic interests align well with the research vision of SQMS, a strong partner of Northwestern University, whose work on qudit-based quantum computing with superconducting radio-frequency (SRF) cavities would benefit from a noise-aware qudit compiler.
	
	In conclusion, I am confident that my academic background and interest in quantum computing will enable me to thrive in this track. I believe the Ph.D.\ program in Computer Science at Northwestern University would challenge me and give me the opportunity to learn, grow, and contribute meaningfully to the future of quantum computing.		
	
	\newpage
	\nocite{*} % show all the references
	%\bibliography{ref}
\end{document}