%%%%%%%%%%%%%%%%%%%%%%%%%%%%%%%%%%%%%%%%%%%%%%%%%%%%
% SPDX-License-Identifier: BSD-3-Clause
% Copyright (c) 2020, Salman Ahmad Faris
% https://github.com/salfaris/EasyPS
% Copyright (c) 2024, Daize Dong
% https://github.com/DaizeDong/Easier-PS-and-SoP
%%%%%%%%%%%%%%%%%%%%%%%%%%%%%%%%%%%%%%%%%%%%%%%%%%%%

\documentclass{easier_sop}

% --- ESSAY DISPLAY SETTINGS ---
\SetStudentName{Leonardo Bove}           % Your name
\SetProgramName{PhD in Computer Science}           % Program you're applying for
\SetUniversityName{Northwestern University}     % University name
\SetUniversityAbbr{\GetUniversityName} % University abbreviation (default as the university name if not set)

% CONTENT
\begin{document}
	\thispagestyle{firstpageheader} % Use the expanded header on the first page
	
	I have always been fascinated by the art of exploiting hardware to perform computation. This passion is the main reason why I chose to attend a technical high school in Italy with a specialization in Electronics, and later pursued a B.Sc.\ in Electronic Engineering at the University of Pisa. For the first time, I was able to fully understand what was really happening beneath my fingers as I typed on a keyboard. I also learned about Moore's law: it was comforting to realize that computers were expected to become faster and more efficient, following a seemingly predictable trajectory. However, I soon discovered that Moore's law had its limits and that the evolution of classical computing was approaching a plateau. I wondered: ``Is that it?'' It felt unbelievable that millennia of human advancement in computational technologies—from the abacus to the Antikythera mechanism—had reached their final stage. Fortunately, I was mistaken. I soon discovered that quantum computing was not a fictional idea but a real and rapidly developing field. This realization is profoundly exciting to me and makes me feel as though we are witnessing a moment as revolutionary as the 1940s, when the first electronic computer, ENIAC, was built.
	
	
	After this discovery, I quickly realized that I wanted to contribute to this scientific and engineering challenge and help make quantum computers useful for humanity. Just as with the classical computing systems we rely on today, their transformative impact became evident only after widespread democratization. I am aware, however, that achieving this for quantum technologies requires not only substantial progress from physicists in developing devices with improved qubit performance, but also the essential contributions of computer scientists and engineers in building the electrical and software infrastructure that makes these systems accessible and usable.

	This is why my academic interests focus on studying and designing new architectures and tools that can help us achieve scalability and the long-anticipated quantum advantage, starting from the Noisy Intermediate-Scale Quantum (NISQ) devices available today. These interests include developing quantum-circuit compilers that are aware of hardware constraints and noise characteristics—thereby enabling optimization and the introduction of error-mitigation techniques—as well as exploring new architectures that integrate classical processing units (e.g., CPUs, GPUs) with Quantum Processing Units (QPUs) to support hybrid algorithms such as the Variational Quantum Eigensolver (VQE). I believe that a Ph.D. program in Computer Science is the most suitable path for addressing these questions: it would provide the academic rigor necessary to engage with this emerging computational revolution and offer invaluable guidance from experienced faculty members.


	As a matter of fact, I believe that a great effort is coming from the research areas of ``Quantum Science and Engineering'' and ``Systems and Networking'' in the Computer Science department of the Northwestern's McCormick School of Engineering. I am particularly intrigued by the research of Professor Kaitlin N. Smith and Professor Nikos Hardavellas, who are actively working on developing compilers that automatically embed quantum error detection codes in high-level quantum circuits and on new modular QPU architectures that give us a viable way of reaching our goals in quantum scaling, together with suitable cross-module hierarchical compilers. Therefore, I would be very interested in pursuing a research activity with them in those areas.
	
	In my view, these academic questions, to which I hope to make meaningful contributions, require a comprehensive understanding of the entire system. Although I did not take a pure Computer Science path for my undergraduate study, I believe I am a strong candidate thanks to my solid background in electronics, complemented by extensive coursework in software theory and computer engineering, as well as my focused experience in quantum computing. I discovered quantum computing at the end of my Bachelor’s degree, and for my thesis I chose to investigate how we control and read out the most widely used superconducting qubit, the transmon, while self-teaching quantum computing and algorithms theory. Through this work, I recognized the fundamental role of classical hardware in quantum systems, which motivated me to pursue a Master’s degree in Electronic Engineering to gain deeper insight into how classical qubit controllers are designed. And I had the opportunity to really dive into that field, thanks to my thesis in collaboration with the Superconducting Quantum Materials and System (SQMS) center at the Fermi National Accelerator Laboratory (Fermilab), where I could assemble my own FPGA based qubit controller using the QICK system. On top of that, I also developed Qubase, a high-level Python pulse sequencer, that allows scientists to write hardware-agnostic experiments and transpile them into hardware specific instructions, after mapping the required resources to those available. This specific experience is what made me understand that I wanted to contribute to the development of this part of the whole infrastructure, the middleware that connects the high-level to the low-level qubit controllers, making quantum computers accessible to a larger group of people.
	
	While I am preparing this application I am simultaneously exploiting my own tools as a graduate research fellow in my thesis supervisor's laboratory in Pisa, where I built the hardware and software infrastructure to make the very first quantum computer of our school of engineering remotely accessible by any interested researcher in the same network, using the intuitive Qubase interface.
	
	I would state with certainty that I will use this PhD degree in Computer Science to pursue a career in the quantum computing industry. On this matter, I believe that Northwestern University, with its campus in Evanston, would be great choice for my career, thanks to its closeness to Chicago, which is rapidly becoming a major hub for quantum computing through large-scale projects, like the the Illinois Quantum and Microelectronics Park and the Chicago Quantum Exchange, and many growing startups, such as Infleqtion. I would benefit from this geographical advantage, thanks to the melting pot of ideas and the dense network of people working towards the same objective, as I would gain significant insights from experts in the field. Furthermore, I believe that my academic interests could fit well with the research vision of SQMS, strong partner of the Northwestern University: their innovative research in qudit-based quantum computing with superconducting radio-frequency (SRF) cavities would certainly benefit from the presence of a qudit noise-aware compiler. I believe that my academic background and my interest in studying quantum computing will allow me to succeed in this track. I am confident that the PhD program in Computer Science at Northwestern University would both challenge me and grant me the opportunity to learn and contribute my ideas to the field of quantum computing.
	
	\newpage
	\nocite{*} % show all the references
	%\bibliography{ref}
\end{document}