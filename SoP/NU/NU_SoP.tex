%%%%%%%%%%%%%%%%%%%%%%%%%%%%%%%%%%%%%%%%%%%%%%%%%%%%
% SPDX-License-Identifier: BSD-3-Clause
% Copyright (c) 2020, Salman Ahmad Faris
% https://github.com/salfaris/EasyPS
% Copyright (c) 2024, Daize Dong
% https://github.com/DaizeDong/Easier-PS-and-SoP
%%%%%%%%%%%%%%%%%%%%%%%%%%%%%%%%%%%%%%%%%%%%%%%%%%%%

\documentclass{easier_sop}

% --- ESSAY DISPLAY SETTINGS ---
\SetStudentName{Leonardo Bove}           % Your name
\SetProgramName{PhD in Computer Science}           % Program you're applying for
\SetUniversityName{Northwestern University}     % University name
\SetUniversityAbbr{\GetUniversityName} % University abbreviation (default as the university name if not set)

% CONTENT
\begin{document}
	\thispagestyle{firstpageheader} % Use the expanded header on the first page
	
	I have always been fascinated by the \emph{art} of exploiting hardware to do computation. That is the main reason why I chose to attend the so-called technical high school in Italy, with a specialization in Electronics, and later on the B.Sc. in Electronic Engineering at the University of Pisa. Finally I was able to fully understand what was \emph{really} going on beneath my fingers as I was typing on my keyboard. I was also exposed to the Moore's law: it was so comforting to understand that computers were destined to perform even faster and better, following a clear predictable path through the years. However, I also had to find out that the Moore's law had an end and, apparently, so did the evolution of human computation. I was thinking "is that it?". I could not believe that the millennial history of advancements in computation technologies made by humans, starting from the abacus and the Antikythera mechanism, had just reached its final evolution. Luckily, I was wrong and soon I discovered that \emph{quantum computing} was not just a fictional term, but rather it was something real and under development. This is unexplainably thrilling for me and makes me feel like we are back in the 1940's when the first electrical computer ENIAC was designed.
	
	After this discovery, I soon realized that I wanted to take part in this scientific and engineering challenge, and make quantum computers \emph{useful} for humanity. As for the classical computation systems we are used to, we could not appreciate a significant impact of those technologies in our lives until their democratization. However, I am aware that this requires not only a significant effort from physicists in conceiving new technologies that improve the performance of \emph{qubits}, but also the unavoidable contribution of computer scientists and engineers in realizing the  electrical and software infrastructure that makes those systems accessible and usable.
	
	That is why my academic interests are primarily in studying and designing new architectures and tools that can help us achieve scalability and the promised \emph{quantum advantage}, starting from the \emph{Noisy Intermediate Scale Quantum} (NISQ) devices that we have available right now \textcolor{red}{cite FTQC}. They include compilers for quantum circuits aware of the available hardware and noise constraints, thus enabling optimization and the introduction of error mitigation codes, and new hybrid architectures that join forces of classical processing units (e.g. CPUs, GPUs) and of QPUs (\emph{Quantum Processing Units}) \textcolor{red}{?}. I believe that a PhD program in Computer Science would be the most suitable way to approach this kind of questions: I would gain from it the required academic rigor needed to successfully tackle this incoming computer revolution in a highly competitive environment, receiving valuable support by experienced faculty members.

	With regard to my academic interests, I believe that a great effort is coming from the research areas of \textit{Quantum Science and Engineering} and \textit{Systems and Networking} in the Computer Science department of the Northwestern's McCormick School of Engineering. I am particularly intrigued by the research of Professor Kaitlin N. Smith and Professor Nikos Hardavellas, who are actively working on developing compilers that automatically embed quantum error detection codes in high-level quantum circuits and on new modular QPU architectures that give us a viable way of reaching our goals in quantum scaling, together with suitable cross-module hierarchical compilers. \textcolor{red}{cite and say that you'd like to work with them for which reasons}.
	
	In my view, these academic questions, to which I hope to make meaningful contributions, require a comprehensive understanding of the entire system. I believe I am a strong candidate thanks to my solid background in electronics, complemented by extensive coursework in software theory and computer engineering, as well as my focused experience in quantum computing. I discovered quantum computing at the end of my Bachelor’s degree, and for my thesis I chose to investigate how we control and read out the most widely used superconducting qubit, the \emph{transmon}. Through this work, I recognized the fundamental role of classical hardware in quantum systems, which motivated me to pursue a Master’s degree in Electronic Engineering to gain deeper insight into how classical qubit controllers are designed. And I had the opportunity to really dive into that field, thanks to my thesis in collaboration with the \emph{SQMS} center at the \emph{Fermi National Accelerator Laboratory} (Fermilab), where I could assemble my own FPGA based qubit controller using the \emph{QICK} system. On top of that, I also developed \emph{Qubase}, a high-level Python pulse sequencer, that allows scientists to write hardware-agnostic experiments and \emph{transpile} them into hardware specific instructions, prior to mapping the required resources to those available. This specific experience is what made me understand that I wanted to contribute to the development of this part of the whole infrastructure, the middleware that connects the high-level to the low-level qubit controllers, making quantum computers accessible to a larger group of people.
	
	While I am preparing this application I am simultaneously exploiting my own tools as a graduate research fellow in my thesis supervisor's laboratory in Pisa, where I built the hardware and software infrastructure to make the very first quantum computer of our school of engineering remotely accessible by any interested researcher in the same network, using the intuitive Qubase interface. \textcolor{red}{eteam}
	
	I would state with certainty that I will use this PhD degree in Computer Science to pursue a career in the quantum computing industry. On this matter, I believe that Northwestern University, with its campus in Evanston, would be great choice for my career, thanks to its closeness to Chicago, which is rapidly becoming a major hub for quantum computing through large-scale projects, like the the \emph{Illinois Quantum and Microelectronics Park} and the \emph{Chicago Quantum Exchange}, and many growing startups, such as \emph{Infleqtion}. I would benefit from this geographical advantage, thanks to the melting pot of ideas and the dense network of people working towards the same objective, as I would gain significant insights from experts in the field. I believe that my academic background and my interest in studying quantum computing will allow me to succeed in this track. I am confident that the PhD program in Computer Science at Northwestern University would both challenge me and grant me the opportunity to learn and contribute my ideas to the field of quantum computing.
	
	
	\textcolor{red}{self taught quantum computing}
	
	
	
	\newpage
	\nocite{*} % show all the references
	%\bibliography{ref}
\end{document}