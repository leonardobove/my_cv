\documentclass[10pt]{article}
\usepackage[english]{babel}
\usepackage{cmbright}
\usepackage{enumitem}
\usepackage{fancyhdr}
\usepackage{fontawesome5}
\usepackage{geometry}
\usepackage{hyperref}
\usepackage[sf]{libertine}
\usepackage{microtype}
\usepackage{paracol}
\usepackage{supertabular}
\usepackage{titlesec}
\usepackage{enumitem}
\hypersetup{colorlinks, urlcolor=black, linkcolor=black}

% Geometry
\geometry{hmargin=1.75cm, vmargin=0.9cm}
\columnratio{0.65, 0.35}
\setlength\columnsep{0.05\textwidth}
\setlength\parindent{0pt}
\setlength{\smallskipamount}{8pt plus 3pt minus 3pt}
\setlength{\medskipamount}{16pt plus 6pt minus 6pt}
\setlength{\bigskipamount}{24pt plus 8pt minus 8pt}

% Style
\pagestyle{empty}
\titleformat{\section}{\scshape\LARGE\raggedright}{}{0em}{}[\titlerule]
\titlespacing{\section}{0pt}{\bigskipamount}{\smallskipamount}
\newcommand{\heading}[2]{\centering{\sffamily\Huge #1}\\\smallskip{\large{#2}}}
\newcommand{\entry}[4]{{{\textbf{#1}}} \hfill #3 \\ #2 \hfill #4}
\newcommand{\tableentry}[3]{\textsc{#1} & #2\expandafter\ifstrequal\expandafter{#3}{}{\\}{\\[6pt]}}

\begin{document}

%\vspace*{\fill}

\begin{paracol}{2}

% Name & headline
\heading{Leonardo Bove}{Graduate Student in\\Electronics Engineering}

\switchcolumn

% Identity card
\vspace{0.01\textheight}
\begin{supertabular}{ll}
  \footnotesize\faPhone & +39 338 449 0477 \\
  \footnotesize\faEnvelope & \href{mailto:leonardo.bove01@gmail.com}{leonardo.bove01@gmail.com} \\
  \footnotesize\faLinkedin & \href{https://www.linkedin.com/in/leonardo-bove-86a1481b3/}{leonardobove} \\
   \footnotesize\faGithub & \href{https://github.com/leonardobove}{leonardobove} \\
\end{supertabular}

\smallskip
\end{paracol}

\section{education}

\entry{MSc in Electronics Engineering}{University of Pisa}{Pisa, Italy}{Sep. 2023 -- present}
\begin{itemize}[noitemsep,leftmargin=3.5mm,rightmargin=0mm,topsep=6pt]
  \item Specialization: Embedded Systems \& Mechatronics
  \item Thesis work: Design of the software and electronic system to control and characterize a superconducting Transmon qubit.
  \item Attended Courses:
    \begin{itemize}[noitemsep]
        \item RF Circuit Design
        \item Microelectronic Fabrication Technologies
        \item Solid State Physics
        \item Digital System Design
        \item Embedded Systems
        \item Sensor And Microsystem Design
        \item Microelectronic System Design
    \end{itemize}
   \item Current GPA: 30/30
   \item Expected graduation date: October 2025
\end{itemize}

\smallskip

\entry{BSc in Electronics Engineering}{University of Pisa}{Pisa, Italy}{Sep. 2020 -- Jul. 2023}
\begin{itemize}[noitemsep,leftmargin=3.5mm,rightmargin=0mm,topsep=6pt]
  \item Thesis title: \textit{Dispersive Readout of  the Transmon Qubit}
  \item Degree grade: 110/110 \textit{Cum Laude}

\end{itemize}

\section{experience}

\entry{Master Thesis Project}{SQMS, Fermilab}{Batavia, IL, USA}{May 2025 -- Jul. 2025}
\begin{itemize}[noitemsep,leftmargin=3.5mm,rightmargin=0mm,topsep=6pt]
	\item Development of \textit{\href{https://github.com/leonardobove/qpcb}{QPCB}}, a custom pulse frequency up/down-conversion board.
	\item Development of Qubase, a qubit pulse sequencer, that relies on the open-source \textit{\href{https://github.com/openquantumhardware/qick}{QICK}} board, a Xilinx FPGA based real-time RF signal generator and readout system, developed at SQMS, Fermilab.
	\item Application of Qubase in 2D and 3D superconducting qubit characterization and other advanced research purposes.  
\end{itemize}

\smallskip

\entry{Chief Technology Officer}{E-Team Squadra Corse, FSAE team}{Pisa, Italy}{Sep. 2023 -- Sep. 2024}
\begin{itemize}[noitemsep,leftmargin=3.5mm,rightmargin=0mm,topsep=6pt]
    \item Define and lead the work of the Electronics and AI \& Software Development divisions
    \item Improve reliability and performance of the electric vehicle 
\end{itemize}

\smallskip

\entry{Embedded Software Developer}{Sintonica s.r.l.}{Navacchio (PI), Italy}{May 2023 -- Sep. 2023}
\begin{itemize}[noitemsep,leftmargin=3.5mm,rightmargin=0mm,topsep=6pt]
    \item Develop the driver firmware for TFT LCD displays for a custom embedded OS on Infineon PSoC ARM microcontroller
    \item Layout of the new release of the company's development kit PCB, based on Cypress and nRF PSoC.
\end{itemize}

\smallskip

\entry{PCB Designer and Embedded System Developer}{E-Team Squadra Corse, FSAE team}{Pisa, Italy}{Sep. 2022 -- Sep. 2023}
\begin{itemize}[noitemsep,leftmargin=3.5mm,rightmargin=0mm,topsep=6pt]
    \item Lead the development of the embedded software of the mounted PCB boards
    \item Develop part of the Vehicle Control Unit software, based on the FreeRTOS real-time OS
    \item Develop a bootloader via CAN bus, ARM and AVR compatible
    \item Unit and integration testing of firmware
\end{itemize}

\section{skills}
\begin{tabular}{l l}
    \textbf{Development Tools \& OS}& Git, Linux, Windows\\
    \textbf{Programming Languages}& C/C++, Python, Verilog, VHDL, MATLAB, Bash, Assembly\\
    \textbf{Quantum Computing Skills}& Superconducting qubit characterization, Qiskit, QuTip\\
    \textbf{FPGA Design}& Vivado, Quartus, Modelsim\\
    \textbf{Analog Circuit Design}& SPICE, ADS\\
    \textbf{MEMS Design}& COMSOL Multiphysics\\
    \textbf{PCB Design}& KiCad, Altium\\
    \textbf{Microcontrollers Architectures}& AVR, ARM\\
    \textbf{Microcontrollers Coding Platforms}& STM32CubeIDE, Microchip Studio, PSoC Creator, Simulink Model-Based Design\\
    \textbf{Electronic Skills}& Electronics lab instrumentation, VNA, tin soldering\\
    \textbf{CAD skills}& 3D printing, Autodesk Fusion 360
\end{tabular}

\section{languages}
\begin{tabular}{l l}
    \textbf{Italian}& Native speaker\\
    \textbf{English}& C2 level\\
    \textbf{German}& B2 level\\
    \textbf{French}& A2 level
\end{tabular}

\section{projects}
\begin{itemize}[noitemsep,leftmargin=3.5mm,rightmargin=0mm,topsep=6pt]
    \item \textbf{\href{https://github.com/leonardobove/spacefiber_pll}{SpaceFibre PLL}}: Model and ADS simulation of a SpaceFibre compatible 6.25GHz PLL, implemented using the SG25H4 0.25 \(\mu\)m SiGe BiCMOS technology from GlobalFoundries.
    \item \textbf{\href{https://github.com/leonardobove/handwritten_digit_recognition}{Handwritten Digit Recognition}}: An handwritten digit recognition system based on a neural network implemented on Altera DE10-Lite board (Altera MAX10 10M50DAF484C7G FPGA)
    \item \textbf{\href{https://github.com/leonardobove/dual_axis_accelerometer}{Dual Axis Accelerometer}}: COMSOL simulation of a dual-axis MEMS accelerometer with T-shape beams.
    \item \textbf{\href{https://github.com/leonardobove/rubiks_cube_automatic_solver/tree/main}{Rubik's Cube Automatic Solver}}: A servo motor actuated Rubik cube solver robot, controlled by S32K144EVB. Developed using Simulink MBD
\end{itemize}

\section{other skills}
\begin{tabular}{l l}
    \textbf{Driving license}& B license\\
    \textbf{Music}& Violin\\
\end{tabular}
\vspace*{\fill}

\end{document}