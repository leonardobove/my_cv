\documentclass[10pt]{article}
\usepackage[english]{babel}
%\usepackage{cmbright}
\usepackage{enumitem}
\usepackage{fancyhdr}
\usepackage{fontawesome5}
\usepackage{geometry}
\usepackage{hyperref}
%\usepackage[sf]{libertine}
\usepackage{microtype}
\usepackage{paracol}
\usepackage{supertabular}
\usepackage{titlesec}
\usepackage{enumitem}
\hypersetup{colorlinks, urlcolor=black, linkcolor=black}

% Geometry
\geometry{hmargin=1.75cm, vmargin=0.9cm}
\columnratio{0.65, 0.35}
\setlength\columnsep{0.05\textwidth}
\setlength\parindent{0pt}
\setlength{\smallskipamount}{8pt plus 3pt minus 3pt}
\setlength{\medskipamount}{16pt plus 6pt minus 6pt}
\setlength{\bigskipamount}{24pt plus 8pt minus 8pt}

% Style
\pagestyle{empty}
\titleformat{\section}{\scshape\LARGE\raggedright}{}{0em}{}[\titlerule]
\titlespacing{\section}{0pt}{\bigskipamount}{\smallskipamount}
%\newcommand{\heading}[2]{\centering{\sffamily\Huge #1}\\\smallskip{\large{#2}}}
\newcommand{\heading}[2]{\centering{\Huge #1}\\\smallskip{\large{#2}}}
\newcommand{\entry}[4]{{{\textbf{#1}}} \hfill #3 \\ #2 \hfill #4}
\newcommand{\tableentry}[3]{\textsc{#1} & #2\expandafter\ifstrequal\expandafter{#3}{}{\\}{\\[6pt]}}

\begin{document}

%\vspace*{\fill}

\begin{paracol}{2}

% Name & headline
\heading{Leonardo Bove}{Graduate Student in\\Electronic Engineering}

\switchcolumn

% Identity card
\vspace{0.01\textheight}
\begin{supertabular}{ll}
  \footnotesize\faPhone & +39 338 449 0477 \\
  \footnotesize\faEnvelope & \href{mailto:leonardo.bove01@gmail.com}{leonardo.bove01@gmail.com} \\
  \footnotesize\faLinkedin & \href{https://www.linkedin.com/in/leonardo-bove-86a1481b3/}{leonardobove} \\
   \footnotesize\faGithub & \href{https://github.com/leonardobove}{leonardobove} \\
\end{supertabular}

\smallskip
\end{paracol}

\section{education}

\entry{M.Sc. in Electronic Engineering}{University of Pisa}{Pisa, Italy}{Sep. 2023 -- Oct. 2025}
\begin{itemize}[noitemsep,leftmargin=3.5mm,rightmargin=0mm,topsep=6pt]
  \item Specialization: Embedded Systems \& Mechatronics
  \item Thesis title: \textit{Superconducting qubit readout and control system based on FPGA and development of a pulse sequencer}
  \item Attended Courses:
    \begin{itemize}[noitemsep]
        \item RF Circuit Design
        \item Microelectronic Fabrication Technologies
        \item Solid State Physics
        \item Digital System Design
        \item Embedded Systems
        \item Sensor And Microsystem Design
        \item Microelectronic System Design
    \end{itemize}
   \item Degree grade: 110/110 with Honors
   \item Overall GPA: 4.0/4.0
\end{itemize}

\smallskip

\entry{B.Sc. in Electronic Engineering}{University of Pisa}{Pisa, Italy}{Sep. 2020 -- Jul. 2023}
\begin{itemize}[noitemsep,leftmargin=3.5mm,rightmargin=0mm,topsep=6pt]
  \item Thesis title: \textit{Dispersive Readout of  the Transmon Qubit}
  \item Degree grade: 110/110 with Honors
  \item Overall GPA: 4.0/4.0

\end{itemize}

\section{research experience}

\entry{Quantum Characterization Graduate Researcher}{Department of Information Engineering, University of Pisa}{Pisa, Italy}{Sep. 2025 -- Present}
\begin{itemize}[noitemsep,leftmargin=3.5mm,rightmargin=0mm,topsep=6pt]
    \item Characterization of multiple qubits using frequency multiplexing techniques.
    \item Software and hardware management of the dilution refrigerator of the laboratory.
    \item Microwave path engineering: sizing and selection of the components of the system; characterization of the system through RF instrumentation (VNA, spectrum analyzer).
    \item FPGA development for control and readout purposes.
    \item Setup of infrastructure for remote qubit characterization (\textit{JupyterLab server}).
\end{itemize}

\entry{Master Thesis Project}{SQMS, Fermilab}{Batavia, IL, USA}{May 2025 -- Jul. 2025}
\begin{itemize}[noitemsep,leftmargin=3.5mm,rightmargin=0mm,topsep=6pt]
    \item Development of \textit{\href{https://github.com/leonardobove/qpcb}{QPCB}}, a custom pulse frequency up/down-conversion board. Microwave EM circuit analysis (\textit{ADS Keysight})
    \item Development of Qubase, a high-level qubit pulse sequencer in Python, that relies on the open-source \textit{\href{https://github.com/openquantumhardware/qick}{QICK}} board, a Xilinx FPGA based real-time RF signal generator and readout system, developed at SQMS, Fermilab.
    \item Application of Qubase in 2D and 3D superconducting qubit characterization and other advanced research purposes.
    \item Automation of the calibration process.
\end{itemize}

\section{industry experience}

\entry{Chief Technology Officer}{E-Team Squadra Corse, FSAE team}{Pisa, Italy}{Sep. 2023 -- Sep. 2024}
\begin{itemize}[noitemsep,leftmargin=3.5mm,rightmargin=0mm,topsep=6pt]
    \item Lead the development of the EV (Electric Vehicle) car of the University of Pisa (\textit{Electronics \& Software Divisions}).
    \item Final approval of projects.
    \item Main high voltage battery manager.
    \item Manager of the GitLab server of the Team.
    \item CI automation manager; PR and issues reviewer.
    \item Team building and teamwork promoter.
    \item Management of people's activities and trips.
\end{itemize}

\smallskip

\entry{Embedded Software Developer}{Sintonica s.r.l.}{Navacchio (PI), Italy}{May 2023 -- Sep. 2023}
\begin{itemize}[noitemsep,leftmargin=3.5mm,rightmargin=0mm,topsep=6pt]
    \item Develop the driver firmware for TFT LCD displays for a custom embedded OS on Infineon PSoC ARM microcontroller
    \item Layout of the new release of the company's development kit PCB, based on Cypress and nRF PSoC.
\end{itemize}

\smallskip

\entry{PCB Designer and Embedded System Developer}{E-Team Squadra Corse, FSAE team}{Pisa, Italy}{Sep. 2022 -- Sep. 2023}
\begin{itemize}[noitemsep,leftmargin=3.5mm,rightmargin=0mm,topsep=6pt]
    \item Lead the development of the embedded software of the mounted PCB boards
    \item Develop part of the Vehicle Control Unit software, based on the FreeRTOS real-time OS
    \item Develop a bootloader via CAN bus, ARM and AVR compatible
    \item Unit and integration testing of firmware
\end{itemize}

\section{skills}
\begin{minipage}[t]{0.48\textwidth}
    \textbf{Development Tools \& OS} \\
    Git, Linux, Windows
    \\[4pt]
    \textbf{Programming Languages} \\
    C/C++, Python, Verilog, VHDL, MATLAB, Bash, Assembly \\[4pt]
    \textbf{Quantum Computing} \\
    Superconducting qubit characterization, Qiskit, QuTiP \\[4pt]
    \textbf{FPGA Design} \\
    Vivado, Quartus, ModelSim
    \\[4pt]
    \textbf{Embedded Systems} \\
    STM32CubeIDE, PSoC Creator, Simulink Model-Based Design
\end{minipage}
\hfill
\begin{minipage}[t]{0.48\textwidth}
    \textbf{Analog Design} \\
    SPICE, ADS, KiCad, Altium
    \\[4pt]
    \textbf{MEMS Design \& Simulation} \\
    COMSOL Multiphysics \\[4pt]
    \textbf{Microcontrollers Architectures} \\
    AVR, ARM
    \\[4pt]
    \textbf{Lab \& Fabrication} \\
    VNA, soldering, prototyping, Autodesk Fusion 360, 3D printing
\end{minipage}

\section{languages}
\begin{tabular}{l l}
    \textbf{Italian} & Native speaker \\
    \textbf{English} & C2 level (Cambridge Assessment English, overall score of 200 equivalent to IELTS 8.5, 2020) \\
    \textbf{German} & B2 level (CLI University of Pisa, 2022) \\
    \textbf{French} & A2 level
\end{tabular}


\section{Grad Course Projects}

\entry{\href{https://github.com/leonardobove/spacefibre_pll}{SpaceFibre PLL}}{University of Pisa, \textit{Wireless Integrated Circuits}, Prof. Daniele Rossi}{Pisa, Italy}{Sep. 2022 -- Sep. 2023}
\begin{itemize}[noitemsep,leftmargin=3.5mm,rightmargin=0mm,topsep=6pt]
    \item Model and ADS simulation of a SpaceFibre compatible 6.25GHz PLL, implemented using the SG25H4 0.25 \(\mu\)m SiGe BiCMOS technology from GlobalFoundries.
    \item Study of the system stability and performance.
\end{itemize}

\smallskip

\entry{\href{https://github.com/leonardobove/handwritten_digit_recognition}{Handwritten Digit Recognition}}{University of Pisa, \textit{Digital System Design}, Prof. Roberto Saletti}{Pisa, Italy}{Jan. 2025 -- Feb. 2025}
\begin{itemize}[noitemsep,leftmargin=3.5mm,rightmargin=0mm,topsep=6pt]
    \item An handwritten digit recognition system based on a neural network implemented on Altera DE10-Lite board (Altera MAX10 10M50DAF484C7G FPGA).
    \item Digital twin implemented in Python for training and quantization.
    \item Custom Verilog driver for touchscreen and visualization on Terasic LT24 screen.
\end{itemize}

\smallskip

\entry{\href{https://github.com/leonardobove/dual_axis_accelerometer}{Dual Axis Accelerometer}}{University of Pisa, \textit{Sensor and Microsystem Design}, Prof. Massimo Piotto}{Pisa, Italy}{Dec. 2024 -- Jan. 2025}
\begin{itemize}[noitemsep,leftmargin=3.5mm,rightmargin=0mm,topsep=6pt]
    \item COMSOL simulation of a dual-axis MEMS accelerometer with T-shape beams.
    \item Study of the device performance and application range, compared to the lumped elements schematic model.
\end{itemize}

\smallskip

\entry{\href{https://github.com/leonardobove/rubiks_cube_automatic_solver/tree/main}{Rubik's Cube Automatic Solver}}{University of Pisa, \textit{Mechatronic Systems Design}, Prof. Roberto Di Rienzo}{Pisa, Italy}{Nov. 2024 -- Dec. 2024}
\begin{itemize}[noitemsep,leftmargin=3.5mm,rightmargin=0mm,topsep=6pt]
    \item A servo motor actuated Rubik cube solver robot, controlled by S32K144EVB
    \item Developed using Simulink Model-Based Design.
    \item Digital twin of the overall system.
\end{itemize}

\section{other skills}
\begin{tabular}{l l}
    \textbf{Driving license}& B license\\
    \textbf{Music}& Violin\\
\end{tabular}
\vspace*{\fill}

\end{document}