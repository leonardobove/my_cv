\documentclass[10pt,stdletter,dateno]{newlfm}
\usepackage{kpfonts}
\usepackage{url}
\usepackage{ragged2e}

\widowpenalty=1000
\clubpenalty=1000

%\noHeadline
%\noFootline

\newlfmP{headermarginskip=20pt}
\newlfmP{sigsize=-50pt}
\newlfmP{dateskipafter=20pt}
\newlfmP{addrfromphone}
\newlfmP{addrfromemail}
\PhrPhone{Phone}
\PhrEmail{Email}

\namefrom{Leonardo Bove}
\addrfrom{%
    \today\\[10pt]
    Dipartimento di Ingegneria dell'Informazione\\
    Università di Pisa\\
    Via Girolamo Caruso, 16, 56122\\
    Pisa, Italy
}
\phonefrom{+393384490477}
\emailfrom{l.bove3@studenti.unipi.it}

\addrto{%
    Christian K. Andersen\\
    QuTech Academy Lead\\
    QuTech, Andersen Lab\\
    Lorentzweg 1, 2628 CJ Delft}

\greetto{Dear Dr. Andersen,}
\closeline{Yours sincerely,}
\begin{document}
    \begin{newlfm}
        
        I am writing to express my interest in the PhD position in Superconducting Qubit Research at the Andersen Lab. I am currently completing my MSc in Electronics Engineering at the University of Pisa and I will be graduating this summer. My academic and research pursuits have been strongly oriented toward quantum technology, with a particular focus on superconducting qubits — a passion that I am eager to further develop at QuTech and TU Delft.
        
        Throughout my academic activities, I have spent significant effort and time in studying and working with quantum systems. My BSc thesis, "\textit{Dispersive Readout of the Transmon Qubit}," laid the groundwork for my keen interest in quantum readout techniques. Currently I am working on my MSc thesis on the design and implementation of a FPGA based control and characterization system for a superconducting Transmon qubit. This project, along with my research on Qubase — a superconducting qubit pulse sequencer developed in collaboration with the QICK project at Fermilab — has given me hands-on experience in dealing with the complexities of quantum state control and manipulation. 
        
        Aside from these specialized projects, I have gained solid technical expertise in quantum computing from my work with platforms such as Qiskit. My research work in superconducting qubit characterization and pulse sequence design has not only enhanced my understanding of quantum coherence but also provided me with a solid analytical framework with which to tackle the issue of scalability of quantum devices. Moreover, my background in electronics engineering gives me a strong knowledge in nanoscale device integration technologies and RF integrated circuit design.
        
        In addition to my technical background, I have cultivated personal qualities that I believe are essential for success in a PhD program. My academic and professional experiences have instilled in me a strong sense of discipline, creativity, and perseverance. As Chief Technology Officer for a student FSAE team, I led the electronics and AI divisions, coordinating complex projects and navigating challenges that demanded both analytical problem-solving and collaborative teamwork. These experiences have not only refined my technical abilities but have also taught me the value of leadership, clear communication, collaboration and resilience when faced with setbacks.
        
        My motivation in pursuing a career in quantum research has always led me to take part in related projects and dive into this new, but promising, field. Now, that same motivation pushes me to be an active member of the research effort and to further develop our knowledge in this area, so to make this technology more accessible and reliable. The decision to pursue a PhD is motivated by my desire to make a lasting impact on the field of quantum technology. I am driven by an insatiable curiosity and a commitment to advancing scientific knowledge, and I see a PhD as the perfect avenue to step deeper into research.
        
        I am particularly drawn to QuTech's superconducting qubit research - especially on large-scale multi-qubit devices - and the potential to work in a very collaborative, interdisciplinary environment. I believe that being part of this vibrant academic community will not only nurture my research aspirations but also allow me to grow as a scientist and contribute meaningfully to the global effort to realize practical quantum computing systems. I think my technical background in quantum measurements, as well as my electronics engineering and embedded systems experience, positions me well to contribute to the research goal at the Andersen Lab.
        
        Thank you for your consideration. I welcome the chance to further explore how my experience can assist in the research being undertaken at QuTech.
        
    \end{newlfm}
\end{document}