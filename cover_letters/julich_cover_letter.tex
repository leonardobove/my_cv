\documentclass[12pt,stdletter,dateno]{newlfm}
\usepackage{kpfonts}
\usepackage{url}

\widowpenalty=1000
\clubpenalty=1000

\noHeadline
\noFootline

\newlfmP{headermarginskip=-50pt}
\newlfmP{sigsize=-50pt}
\newlfmP{dateskipafter=20pt}
\newlfmP{addrfromphone}
\newlfmP{addrfromemail}
\PhrPhone{Phone}
\PhrEmail{Email}

\namefrom{Leonardo Bove}
\addrfrom{%
    \today\\[10pt]
    Dipartimento di Ingegneria dell'Informazione\\
    Università di Pisa\\
    Via Girolamo Caruso, 16, 56122\\
    Pisa, Italy
}
\phonefrom{+393384490477}
\emailfrom{l.bove3@studenti.unipi.it}


%\addrto{%
%    Faculty Search Committee\\
%    Department of Computer Science\\
%    Clemson University\\
%    Clemson, SC 29634-0974}

\greetto{To Whom It May Concern,}
\closeline{Yours Faithfully,}
\begin{document}
    \begin{newlfm}
        
        I am writing to express my keen interest in the PhD position in the development of scalable superconducting quantum systems at Forschungszentrum Jülich. I am currently completing my MSc in Electronics Engineering at the University of Pisa, specializing in Embedded Systems \& Mechatronics, and will be graduating by the end of this summer. My academic journey has been deeply intertwined with quantum research — with a particular focus on superconducting qubits — and I am eager to contribute to Jülich’s efforts in advancing scalable quantum computing technologies. I would be available to start work at Jülich from October 2025.
        
        My research experience includes designing and implementing an FPGA-based control and characterization system for superconducting Transmon qubits, as well as contributing to Qubase — a PYNQ-based superconducting qubit pulse sequencer developed in collaboration with the QICK project at Fermilab. These projects, along with coursework in RF Circuit Design, Microelectronic Fabrication Technologies, and Solid State Physics, have given me a strong foundation in quantum hardware, cryogenic measurements, and signal processing. Moreover, my expertise in FPGA development, Verilog, Python, and embedded systems aligns well with the development of quantum control stacks.
        
        In addition to my technical background, I have cultivated personal qualities that I believe are essential for success in a PhD program. My academic and professional experiences have instilled in me a strong sense of discipline, creativity and perseverance. As Chief Technology Officer for a student FSAE team, I led the electronics and AI divisions, coordinating complex projects and navigating challenges that demanded both analytical problem-solving and collaborative teamwork. These experiences have not only refined my technical abilities but have also taught me the value of leadership, clear communication, collaboration and resilience when faced with setbacks.
        
        Looking forward, I am particularly interested in contributing to the development of scalable superconducting quantum devices, including microwave control, cryogenic characterization, and qubit calibration at Forschungszentrum Jülich. My decision to pursue a PhD stems from the profound impact I believe quantum computing will have across various scientific disciplines, such as atomic-scale simulations, much like classical computing has revolutionized scientific methodologies. Given the current state of the field, superconducting qubits represent one of the most promising approaches to realizing practical quantum computation, and I am keen to be part of the research efforts driving this progress. I am particularly interested in developing novel strategies for quantum error correction and noise reduction in superconducting circuits — a field where I believe my background in both theoretical principles and practical system design will be especially valuable.
        
        Thank you for considering my application. I look forward to the opportunity to discuss how my background and passion for superconducting quantum systems align with the research goals of Forschungszentrum Jülich.
        
    \end{newlfm}
\end{document}