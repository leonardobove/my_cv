\documentclass[12pt,stdletter,dateno]{newlfm}
\usepackage{kpfonts}
\usepackage{url}

\widowpenalty=1000
\clubpenalty=1000

\noHeadline
\noFootline

\newlfmP{headermarginskip=-50pt}
\newlfmP{sigsize=-50pt}
\newlfmP{dateskipafter=20pt}
\newlfmP{addrfromphone}
\newlfmP{addrfromemail}
\PhrPhone{Phone}
\PhrEmail{Email}

\namefrom{Leonardo Bove}
\addrfrom{%
    \today\\[10pt]
    Dipartimento di Ingegneria dell'Informazione\\
    Università di Pisa\\
    Via Girolamo Caruso, 16, 56122\\
    Pisa, Italy
}
\phonefrom{+393384490477}
\emailfrom{l.bove3@studenti.unipi.it}


\addrto{%
    Dr. Peter Schüffelgen\\
    Peter Grünberg Institut, Halbleiter-Nanoelektronik (PGI-9)\\
    Forschungszentrum Jülich GmbH\\
    Wilhelm-Johnen-Straße\\
    52428 Jülich}

\greetto{Dear Dr. Schüffelgen,}
\closeline{Yours Sincerely,}
\begin{document}
    \begin{newlfm}
        
       I am writing to express my keen interest in the PhD position focused on the development of scalable superconducting quantum systems at Forschungszentrum Jülich. I am currently completing my MSc in Electronics Engineering at the University of Pisa, specializing in Embedded Systems \& Mechatronics, and will be graduating by the end of this summer. I have cultivated a strong foundation in quantum research, particularly in superconducting qubits, and I am excited about the opportunity to contribute to your research group’s innovative work in topological and superconducting quantum devices.
       
       During my academic journey, I have been actively engaged in quantum research projects. My master thesis work includes designing and implementing an FPGA-based control and characterization system for superconducting Transmon qubits, as well as contributing to the development of Qubase—a PYNQ-based superconducting qubit pulse sequencer in collaboration with the QICK project at Fermilab. These experiences, combined with advanced coursework in RF Circuit Design, Microelectronic Fabrication Technologies, and Solid State Physics, have equipped me with in-depth knowledge of quantum hardware, cryogenic measurements, and signal processing.
       
       I am also proud to highlight my leadership experience as the Chief Technology Officer for a student FSAE team, where I led both the electronics and AI divisions. This role required not only a high level of technical expertise but also strong communication, collaboration, and problem-solving skills—qualities I believe are essential for success in a PhD program and in advancing research at your group.
       
       Looking ahead, I am particularly drawn to your group's pioneering work in developing superconducting and topological quantum devices—especially your efforts to integrate topological insulators with superconducting circuits to build robust quantum hardware. My decision to pursue a PhD stems from the profound impact I believe quantum computing will have across various scientific disciplines, such as atomic-scale simulations, much like classical computing has revolutionized scientific methodologies. Given the current state of the field, topological and superconducting qubits represent two of the most promising approaches to realizing practical quantum computation, and I am keen to be part of the research efforts driving this progress. Leveraging my background in both theoretical principles and practical system design, I am eager to develop novel approaches for quantum error correction and noise reduction specifically tailored for topological quantum chips. I am convinced that my expertise and passion will be a strong asset in supporting and advancing the groundbreaking research conducted under your guidance.
       
       I would be honored to contribute to the exciting research efforts at your group. Thank you for considering my application. I look forward to the opportunity to discuss in further detail how my background and research interests align with the goals of your team.
        
    \end{newlfm}
\end{document}