\documentclass[12pt,stdletter,dateno]{newlfm}
\usepackage{kpfonts}
\usepackage{url}

\widowpenalty=1000
\clubpenalty=1000

\noHeadline
\noFootline

\newlfmP{headermarginskip=10pt}
\newlfmP{sigsize=-50pt}
\newlfmP{dateskipafter=20pt}
\newlfmP{addrfromphone}
\newlfmP{addrfromemail}
\PhrPhone{Phone}
\PhrEmail{Email}

\namefrom{Leonardo Bove}
\addrfrom{%
    \today\\[10pt]
    Dipartimento di Ingegneria dell'Informazione\\
    Università di Pisa\\
    Via Girolamo Caruso, 16, 56122\\
    Pisa, Italy
}
\phonefrom{+393384490477}
\emailfrom{l.bove3@studenti.unipi.it}


%\addrto{%
%    Faculty Search Committee\\
%    Department of Computer Science\\
%    Clemson University\\
%    Clemson, SC 29634-0974}

\greetto{To Whom It May Concern,}
\closeline{Yours Faithfully,}
\begin{document}
    \begin{newlfm}
        
        I am writing to express my strong interest in the PhD position in Quantum Computing with Superconducting Circuits at ETH Zurich, as advertised. I am currently completing my MSc in Electronics Engineering at the University of Pisa and will be graduating this summer. My academic journey has been deeply intertwined with quantum research—particularly in the experimental aspects of superconducting qubits—and I am eager to further develop my expertise in ETH Zurich’s dynamic, interdisciplinary research environment.
        
        During my BSc studies, my thesis titled “\textit{Dispersive Readout of the Transmon Qubit}” ignited my passion for quantum measurement techniques. Building on that foundation, my MSc thesis involves designing and implementing an FPGA-based control and characterization system for superconducting Transmon qubits. In addition, my work on Qubase—a superconducting qubit pulse sequencer developed in collaboration with the QICK project at Fermilab—has provided me with practical, hands-on experience in quantum state control and experimental system design.
        
        Alongside these specialized projects, I have honed my technical expertise in quantum computing platforms, such as Qiskit, and have gained valuable insights into quantum error correction, noise reduction, and scalability challenges. My solid background in electronics engineering has equipped me with in-depth knowledge of nanoscale device integration and RF integrated circuit design—skills that are directly relevant to the technical demands of scaling up superconducting qubit systems.
        
        Beyond my technical qualifications, I bring strong personal qualities essential for research success, including discipline, creativity, and perseverance. My role as Chief Technology Officer for a student FSAE team allowed me to lead the electronics and AI divisions, coordinating complex projects and refining my leadership, problem-solving, and communication skills. I am confident that these experiences, coupled with my technical background, position me well to contribute to ETH Zurich’s efforts in developing fault-tolerant quantum computing systems.
        
        Looking forward, I am particularly interested in contributing to the development of scalable superconducting quantum devices, including microwave control and qubit calibration at Quantum Device Lab. My decision to pursue a PhD stems from the profound impact I believe quantum computing will have across various scientific disciplines, such as atomic-scale simulations, much like classical computing has revolutionized scientific methodologies. Given the current state of the field, superconducting qubits represent one of the most promising approaches to realizing practical quantum computation, and I am keen to be part of the research efforts driving this progress. I am particularly interested in developing novel strategies for quantum error correction and noise reduction in superconducting circuits — a field where I believe my background in both theoretical principles and practical system design will be especially valuable.
        
        I am particularly excited by the opportunity to work within ETH Zurich’s Quantum Device Lab. I look forward to contributing to the development of automated testing and characterization techniques for large-scale superconducting circuits, and to advancing novel strategies in quantum error correction and experiment design. I am convinced that my background in both theoretical principles and practical system design will be a valuable asset in pushing the boundaries of experimental quantum computing at ETH Zurich.
        
        The professors who are guiding me in my master thesis work would be glad to serve as a reference for my profile: Professor Massimo Macucci (\texttt{massimo@mercurio.iet.unipi.it}) (who sent his reference letter directly by e-mail to \texttt{wallraff.office@phys.ethz.ch}), Professor Stefano Di Pascoli (\texttt{stefano.di.pascoli@unipi.it}) and Professor Massimo Piotto (\texttt{massimo.piotto@unipi.it}).
        
        Thank you for considering my application. I look forward to the opportunity to discuss how my background, passion, and vision align with the goals of ETH Zurich’s Quantum Device Lab.
        
    \end{newlfm}
\end{document}